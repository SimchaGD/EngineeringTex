\documentclass[11pt]{article}
\usepackage{amsmath}
\usepackage[makeroom]{cancel}
\usepackage[framed, numbered, autolinebreaks, useliterate]{mcode}
\begin{document}
\title{Engineering week 3 opgaven}
\author{Simcha van Helvoort}
\maketitle

Dit bestand bevat de samenvatting en de antwoorden van Engineering week 3 voor het oefenen van mijn LaTeX skills. Eerst zal wordt de stof van deze week uitgelegd en daarna zal ik mijn uitwerkingen van het huiswerk publiceren.

\newpage

\section{Voorbeeld}

$$-\dfrac{d^2u}{dx^2} = 1 = f(x) = f_{\Omega}$$ met $x \in (0, 1) \\
$
Met randvoorwaarden:
\begin{align*}
    u(0) = 1\\
	\dfrac{du}{dx}\Big|_{x = 1} = \alpha = 0
\end{align*}

We benaderen $u(x)$ met de centrale differentie techniek.\\
Dit is de vergelijking voor $n = 5$.\\
We beginnen met de basis vergelijking en gaan stap voor stap naar de \\vergelijking die hetzelfde als ODE representeerd:\\\\
\begin{gather*}
\begin{pmatrix}
-2& 1 & 0 & 0 & 0 \\
1 & -2& 1 & 0 & 0 \\
0 & 1 & -2& 1 & 0 \\
0 & 0 & 1 & -2& 1 \\
0 & 0 & 0 & 1& -2 
\end{pmatrix}
%
\begin{pmatrix}
u_{1}\\
u_{2}\\
u_{3}\\
u_{4}\\
u_{5}\\
\end{pmatrix}
=
-h^2
\begin{pmatrix}
f_{1}\\
f_{2}\\
f_{3}\\
f_{4}\\
f_{5}\\
\end{pmatrix}\\
u(0)= u_{1} = 1 
\end{gather*}
Dit doen we op dezelfde manier als bij de Dirichletvoorwaarden, maar wat moet komen te staan op de plek van de Neumann randvoorwaarden?\\\\
\begin{gather*}
\begin{pmatrix}
1 & 0 & 0 & 0 & 0 \\
1 & -2& 1 & 0 & 0 \\
0 & 1 & -2& 1 & 0 \\
0 & 0 & 1 & -2& 1 \\
 &   & ? & &  
\end{pmatrix}
%
\begin{pmatrix}
u_{1}\\
u_{2}\\
u_{3}\\
u_{4}\\
u_{5}\\
\end{pmatrix}
=
\begin{pmatrix}
1\\
-h^2f_{2}\\
-h^2f_{3}\\
-h^2f_{4}\\
-h^2f_{5}\\
\end{pmatrix}\\
\dfrac{du_{5}}{dx} = \alpha = 0
\end{gather*}
Dit kan benaderd worden als
$\frac{u_{6}-u_{4}}{2h} = \alpha$.\\
Zoals je weet licht $u_{6}$ buiten je domein, daarom defineren we $u_{6}$ als volgt:\\
$$
\frac{u_{6}-u_{4}}{2h} = \alpha \implies u_{6}-u_{4} = 2h\alpha \implies u_{6} = u_{4} + 2h\alpha
$$
Dus\\
\begin{gather*}
{d^2u}{dx^2} \approx -\frac{1}{h^2}\big(u_{4}-2u_{5}+(u_{4} + 2h\alpha)\big) = f_{5} \implies\\
-\frac{1}{h^2}(2u_{4}-2u_{5}+ 2h\alpha) = f_{5}
\implies (2u_{4}-2u_{5}+ 2h\alpha) = -h^2f_{5} \implies\\
(2u_{4}-2u_{5}) = -h^2f_{5} - 2h\alpha\\
\end{gather*}
Dit geeft de uiteindelijke vergelijking:\\
$$
\begin{pmatrix}
1 & 0 & 0 & 0 & 0 \\
1 & -2& 1 & 0 & 0 \\
0 & 1 & -2& 1 & 0 \\
0 & 0 & 1 & -2& 1 \\
0 & 0 & 0 & 0 & 1 
\end{pmatrix}
%
\begin{pmatrix}
u_{1}\\
u_{2}\\
u_{3}\\
u_{4}\\
u_{5}\\
\end{pmatrix}
=
\begin{pmatrix}
1\\
-h^2f_{2}\\
-h^2f_{3}\\
-h^2f_{4}\\
-h^2f_{5}\\
\end{pmatrix}
+
\begin{pmatrix}
0\\
0\\
0\\
0\\
-2h\alpha
\end{pmatrix}
$$


\section{Opgave 1}
ODE:\\
$$
-\dfrac{d^2u}{dx^2} = 1\\\\
$$
Randvoorwaarden:\\
\begin{align*}
	\dfrac{du(0)}{dx} = 0 \\
				 u(1) = 0
\end{align*}


a: De exacte oplossing\\
\begin{gather*}
\int d^2u=\int dx^2 \implies \dfrac{du}{dx} = -x + C\implies \dfrac{du(0)}{dx} = C = 0\\
\int{du} = \int -x dx \implies u = -\frac{1}{2}x^2 + D \implies u(1) = -\frac{1}{2} + D = 0 \implies D = \frac{1}{2}
\end{gather*}
Het uiteindelijke antwoord:
$$u(x) = -\frac{1}{2}x^2 + \frac{1}{2}$$
b: De numerieke oplossing met $n = 5$.
$$
u(x_{i-1}) - 2u(x_{i}) + u(x_{i+1}) = -h^2f_{\Omega}
$$
\begin{gather*}
\begin{pmatrix}
	  &  & ?&  &  \\
	 1&-2& 1& 0& 0\\
	 0& 1&-2& 1& 0\\
	 0& 0& 1&-2& 1\\
	 0& 0& 0& 0& 1
\end{pmatrix}
\begin{pmatrix}
	u_{1}\\
	u_{2}\\
	u_{3}\\
	u_{4}\\
	u_{5}
\end{pmatrix}=-h^2\vec{f}\\
\dfrac{du_{1}}{dx}\Big|_{x = 0} = \alpha = 0\approx \frac{u_{2}-u_{0}}{2h}
\end{gather*}
Netzo als het voorbeeld drukken we $u_{0}$ uit in $u_{2}$.
$$
u_{0} = -2h\alpha + u_{2}
$$
Daardoor kunnen $u_{1}$ omschrijven naar
$$
u_{1} = \frac{u_{0}-2u_{1}+u_{2}}{h^2} = \frac{-2u_{1}+u_{2}-2h\alpha}{h^2} = -2u_{1} + 2u_{2} - \xcancel{2h\alpha} \text{want } \alpha = 0
$$
De uiteindelijke vergelijking:
$$
\begin{pmatrix}
	-2& 2& 0& 0& 0\\
	 1&-2& 1& 0& 0\\
	 0& 1&-2& 1& 0\\
	 0& 0& 1&-2& 1\\
	 0& 0& 0& 0& 1
\end{pmatrix}
\vec{u} = -h^2
\begin{pmatrix}
1\\
1\\
1\\
1\\
0
\end{pmatrix}
$$
\newpage
Het bijbehorende matlab-script:
\begin{lstlisting}
clear, close, clc

a = 0;
b = 1;
n = 3;
h = (b-a)/(n-1);
v = a:h:b;

% make matrix A
rowToeplitz = zeros(1,n);
rowToeplitz(1) = -2;
rowToeplitz(2) = 1;

coefMAT = toeplitz(rowToeplitz);
du0dxRow = zeros(1,n);
du0dxRow(1) = -2;
du0dxRow(2) = 2;
coefMAT(1,:) = du0dxRow;

u0Row = zeros(1,n);
u0Row(n) = 1;
coefMAT(n,:) = u0Row

% make vector f
f = ones(n,1)*-h^2;
f(n) = 0

U = coefMAT\f

X = linspace(a,b,100);
syms u(x)
equation = -diff(u,x,2) == 1
du = diff(u,x)
BC = [du(0) == 0, u(1) == 0];
uSol(x) = dsolve(equation, BC)
plot(X,uSol(X))
hold on
plot(v,U, '*-')

%%
clear,close,clc

a = 0;
b = 1;
n = 40;
h = (b-a)/(n-1);
v = a:h:b;

% make matrix A
rowToeplitz = zeros(1,n);
rowToeplitz(1) = -2;
rowToeplitz(2) = 1;

coefMAT = toeplitz(rowToeplitz);

du0dxRow = zeros(1,n);
du0dxRow(1) = -2;
du0dxRow(2) = 2;
coefMAT(1,:) = du0dxRow;

u0Row = zeros(1,n);
u0Row(n) = 1;
coefMAT(n,:) = u0Row

% make vector f
f = ones(n,1)*-12*(0.5^2)*(h)^2;
f(n) = 0

U = coefMAT\f
\end{lstlisting}
\newpage
\section{Opdracht 2}
$$-\dfrac{d^2u}{dx^2} = 12x^2$$
$$ x \in (0,1)$$
Met randvoorwaarden: 
\begin{align*}
\dfrac{du(0)}{dx} = 0\\
u(1) = 0
\end{align*}
a: De exacte oplossing
\begin{gather*}
\dfrac{d^2u}{dx^2} = -12x^2 \implies \dfrac{du}{dx} = -4x^3 + C\\
\dfrac{du(0)}{dx} = -4 \times 0^3 + C = 0 \implies C = 0\\
\dfrac{du}{dx} = -4x^3 \implies u(x) = -x^4 + D\\
u(1) = -1^4 + D = 0 \implies D = 1
\end{gather*}

b: De numerieke oplossing


\end{document}